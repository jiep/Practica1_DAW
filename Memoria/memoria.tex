\documentclass[10pt,a4paper]{article}
\usepackage{fontspec}
\usepackage{amsmath}
\usepackage{amsfonts}
\usepackage{amssymb}
\usepackage{graphicx}
\usepackage[spanish]{babel}

\input{portada}

\newcommand*{\autores}{
\begin{tabular}{r l}
GII+GIS: & Germán Alonso Azcutia \\
GIS+MAT: & José Ignacio Escribano Pablos
\end{tabular}
}

\begin{document}

\portada{Práctica 1}{Diseño de Aplicaciones Web}{Página web interactiva: HTML, CSS, jQuery y Bootstrap }{\autores}{Móstoles}

\section{Página 1: CSS, HTML y jQuery}

\subsection{Organización de los archivos}

Lo primero que hicimos fue crear el archivo \texttt{"index\_css.html"} donde se incluye la página web y el archivo \texttt{"style.css"} donde se encuentran los estilos de los elemento de dicha web.

La estructura inicial del documento \texttt{"index\_css.html"} es la siguiente:

\begin{verbatim}
<!DOCTYPE html>

<html>

<head>

<title>Result</title>

<link type="text/css" rel="stylesheet" href="style.css" />

</head>

<body>

</body>

</html>
\end{verbatim}

Es decir, una estructura básica del documento html, en donde lo único destacable es que el archivo está enlazado con una hoja de estilos \texttt{"style.css"}.

\subsection{Estructuración del HTML}

Para organizar la página web hacemos uso de la etiqueta \texttt{"<div>"} para definir los bloques de contenido y secciones.

La estructura con los \texttt{"<div>"} hechos sería la siguiente:
\begin{verbatim}
<!DOCTYPE html>
<html>
<head>
<link type="text/css" rel="stylesheet" href="css/style.css" />
</head>

<body>
<div id="header">
	<div id="logo">
		<img src="css/img/bg_head_top_logo.jpg">
		<h1>Metaheuristics Research Projects</h1>
	</div>
</div>
<div id="header_bottom"></div>
<div id="horizontal_menu">
	<div id="despl_menu1" class="menu_h">Metaheuristics Overview
		<ul class="menu_despl">
	</div>
	<div id="despl_menu2" class="menu_h">
		Problems
	</div>
	<div id="despl_menu3" class="menu_h">
		Black-Box Solvers
	</div>
	<div id="despl_menu4" class="menu_h">
		Multi-objective Problems
	</div>
</div>
<div id="breadcrum">
</div>
<div id="content">
	<div id="vertical_menu">
		<h3>Sections<div id="corner">
	</div>
</div>
<div id="content_right">
</div>
<div id="footer">
		<p>Designed by <emph>José Ignacio Escribano</emph> y <emph>Germán Alonso</emph></p>
</div>
</body>
</html>
\end{verbatim}

Pero, evidentemente esto no esta bien, para conseguir el objetivo de nuestra página, tenemos que utilizar la hoja de estilos y jugar con las propiedades del CSS.

Para poder realizar nuestro proposito, se han utilizado identificadores en los diferentes \texttt{"<div>"}, como \texttt{"<div id="footer"> o <div id="horizontal\_menu">}, para poder diferenciar cada bloque de la web y poder aplicar los estilos deseados.


\subsection{Imagenes y Contenido}

Las imágenes que aparecen en la web, estan dentro de la carpeta \texttt{"img"} y para meterlas dentro de nuestra página hemos usado distintos métodos, según lo necesario para cada una de ellas:

Las imágenes que ocupan todo el ancho de nuestra web, o el ancho que queremos usar, se han metido utilizando la propiedad \texttt{"background-image"}

Hay otras imágenes que son mucho más pequeñas para el tamaño que necesitamos, para ello hemos utilizado la propiedad \texttt{repeat-y} para que dicha imagen se repita en el eje y.

Otras imágenes, como la del logo, tienen un contenedor \texttt{<div>} para ella sola y en donde se mete la imagen directamente en el HTML, sin tener que estar en las propiedades css como las anteriores. 

Sobre el contenido, simplemente hemos rellenado el documento HTML con la información que viene en la página de ejemplo. 

\subsection{Propiedades CSS}

\subsubsection{Propiedades css de los <div>}
En las etiquetas \texttt{<div>}, las propiedades más utilizadas son \texttt{position, width y height}.

\texttt{Width y height} son propiedades que se utilizan para modificar el tamaño de las etiquetas \texttt{"<div>"} y adecuarlo así a las necesidades de la web.

Pero la propiedad más importante es position, con ella conseguimos posicionar los elementos de manera correcta. Los dos valores utilizados en esta práctica son:

%\texttt{"Relative:"} con ella, el navegador reserva el espacio que tendría el elemento dentro del documento, si todo estuviera por defecto. Es decir, el "hueco" de la posición por defecto se mantiene aunque el elemento se desplace.

\texttt{Absolute:} en este caso, el navegador no reserva espacio, por lo que el elemento sería absoluto respecto a su contenedor.

Además de estas propiedades, se han utilizado otras como \texttt{background-color, color, text-align, left, right, top o bottom}, que nos facilitan colocar de manera correcta los elementos en cada bloque.

Un ejemplo puede ser el siguiente:

\begin{verbatim}
#logo > h1{
  position: relative;
  top: -40px;
  right: -70px;
  z-index: 1;
  font-size: 1.5em;
  color: #8f8f8f;
}
\end{verbatim}

\subsubsection{Propiedades css para decoración}

En este apartado, la web ya está prácticamente terminada, solo queda 

\subsection{Menús desplegable con jQuery}

\section{Página 2: CSS, HTML, JS}

\end{document}
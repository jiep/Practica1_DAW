\documentclass[10pt,a4paper]{article}
\usepackage{fontspec}
\usepackage{amsmath}
\usepackage{amsfonts}
\usepackage{amssymb}
\usepackage{graphicx}
\usepackage[spanish]{babel}

\input{portada}

\newcommand*{\autores}{
\begin{tabular}{r l}
GII+GIS: & Germán Alonso Azcutia \\
GIS+MAT: & José Ignacio Escribano Pablos
\end{tabular}
}

\begin{document}

\portada{Práctica 1}{Diseño de Aplicaciones Web}{Página web interactiva: HTML, CSS, jQuery y Bootstrap }{\autores}{Móstoles}

\section{Página 1: CSS, HTML y jQuery}

\subsection{Organización de los archivos}

Lo primero que hicimos fue crear el archivo \texttt{"index\_css.html"} donde se incluye la página web y el archivo \texttt{"style.css"} donde se encuentran los estilos de los elemento de dicha web.\\

La estructura inicial del documento \texttt{"index\_css.html"} es la siguiente:

\begin{verbatim}
<!DOCTYPE html>
<html>
<head>
<title>Result</title>
<link type="text/css" rel="stylesheet" href="style.css" />
</head>
<body>
</body>
</html>
\end{verbatim}

Es decir, una estructura básica del documento html, en donde lo único destacable es que el archivo está enlazado con una hoja de estilos \texttt{"style.css"}.

\subsection{Aspectos relevantes del HTML}

Hemos usado una estructura básica del documento HTML, lo más reelevante es el uso de una hoja de estilos \texttt{"style.css"} y que para organizar la página web hacemos uso de la etiqueta \texttt{"<div>"} para definir los bloques de contenido y secciones.\\

La estructura de los \texttt{"<div>"} hechos sería como la siguiente:

\begin{verbatim}
<div id="breadcrum">
	<img src="css/img/bg_bullet_arrow.gif">
	Home
</div>
\end{verbatim}

\subsection{Aspectos relevantes del CSS}

\subsubsection{Propiedades css de los <div>}

En las etiquetas \texttt{<div>}, las propiedades más utilizadas son \texttt{position, width y height}. Con \texttt{Width y height} hemos modificado el tamaño de las etiquetas \texttt{"<div>"} para adecuarlo así a las necesidades de la web, y con la propiedad \texttt{position} y jugando con sus valores \texttt{Relative} y \texttt{Absolute} conseguimos posicionar los elementos de manera correcta.\\

Además de estas propiedades, se han utilizado otras como \texttt{background-color, color, text-align, left, right, top o bottom}, que nos facilitan colocar de manera correcta los elementos en cada bloque.\\

Un ejemplo puede ser el siguiente:

\begin{verbatim}

#logo > h1{
  position: relative;
  top: -40px;
  right: -70px;
  z-index: 1;
  font-size: 1.5em;
  color: #8f8f8f;
}

\end{verbatim}



\subsubsection{Propiedades css para decoración}



Para decorar y que la página web tenga un aspecto similar a la pedida, se han utilizado propiedades como \texttt{font-weight, font-stretch ,color, border-color, border-radius, margin-left, margin-right, text-align},
que nos permiten cambiar el grosor del texto y alinearlo, poner los márgenes necesarios y adaptar los bordes a nuestro gusto, entre otras cosas.


\subsection{Imagenes}



Las imágenes que aparecen en la web, estan dentro de la carpeta \texttt{"img"} y para meterlas dentro de nuestra página hemos usado distintos métodos, según lo necesario para cada una de ellas:\\


Las imágenes que ocupan todo el ancho de nuestra web, o el ancho que queremos usar, se han metido utilizando la propiedad \texttt{"background-image"}.\\


Hay otras imágenes que son mucho más pequeñas para el tamaño que necesitamos, para ello hemos utilizado la propiedad \texttt{repeat-y} para que dicha imagen se repita en el eje y.\\


Otras imágenes, como la del logo, tienen un contenedor \texttt{<div>} para ella sola y en donde se mete la imagen directamente en el HTML, sin tener que estar en las propiedades css como las anteriores. 



\subsection{Menús desplegable con jQuery}

Para agregar los efectos pedidos (fadein y drop down) en el menú, lo primero que tenemos que hacer es importar los archivos en el HTML.

\begin{verbatim}
 <script src="js/jquery-2.1.3.min.js"></script>
 <script src="js/menu.js"></script>
\end{verbatim}
 
Para realizar el código nos basamos en el método \texttt{animate()}, que nos permite modificar varias propiedades del CSS al mismo tiempo.\\

Gracias a esto, cambiamos los valores \texttt{heigth} y \texttt{opacity} en los eventos de ratón \texttt{mouseenter} y \texttt{mouseleave} para conseguir los efectos deseados.

\section{Página 2: CSS, HTML, JS y BOOTSTRAP}

\subsection{Organización de los archivos}
En este caso, los archivos creados fueron \texttt{"index\_bootstrap.html"} y \texttt{"custom.css"}.
El HEAD de nuestro archivo queda de la siguiente forma:
\begin{verbatim}
  <head>
    <meta charset="utf-8">
    <meta name="viewport" content="width=device-width", initial-scale="1.0" />
    <title>Metaheuristics Research Projects</title>
    <script src="js/jquery-2.1.3.min.js"></script>
    <script src="js/bootstrap.min.js"></script>
    <script src="js/collapse.js"></script>
    <link type="text/css" rel="stylesheet" href="css/bootstrap.min.css"/>
    <link type="text/css" rel="stylesheet" href="css/custom.css"/>
  </head>
\end{verbatim}

Como se puede observar, el CSS creado por nosotros debe estar aparecer debajo de la de bootstrap para que cuando los estilos entren en conflicto, prevalezca el nuestro.

\subsection{Aspectos relevantes del HTML}

La estructura de \texttt{"index\_bootstrap.html"} es bastante similar al 
\end{document}